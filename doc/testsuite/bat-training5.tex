\documentclass[11pt]{beamer}

\usepackage{url}
\usepackage{tikz}
%\author{Armijn Hemel}
\title{Using the Binary Analysis Tool - part 5}
\date{}

\begin{document}

\setlength{\parskip}{4pt}

\frame{\titlepage}

\frame{
\frametitle{Subjects}
In this course you will learn:

\begin{itemize}
\item adding new identifiers to BAT
\item writing a new prerun scan
\item writing a new unpack scan
\item writing a new file specific scan
\item writing a new postrun scan
\end{itemize}
}

\frame{
\frametitle{Adding identifiers to identifier search}
Identifiers for files are stored in \texttt{bat/fsmagic.py}. To add an identifier:

\begin{enumerate}
\item add it to the \texttt{fsmagic} dictionary
\item add an offset to the \texttt{correction} dictionary if the identifier does not start at byte 0
\item optionally group identifiers in an array, similar to \texttt{squashtypes}
\end{enumerate}

A new identifier will be scanned as soon as there is a scan that has declared it in the configuration in the \texttt{magic} option for the scan.
}

\begin{frame}[fragile]
\frametitle{Using identifiers in scans}
Identifiers can be accessed in a dictionary called \texttt{offsets}. This dictionary is passed around as an argument
 to each scan and can be accessed by the keys from the \texttt{fsmagic} dictionary, for example:

\begin{verbatim}
if offsets['java_serialized'] == []:
        return ([], blacklist, [])
\end{verbatim}
\end{frame}

\begin{frame}[fragile]
\frametitle{Using identifiers in scans}
\end{frame}

\frame{
\frametitle{Conclusion}
In this course you have learned about:

\begin{itemize}
\item adding new identifiers to BAT
\item writing a new prerun scan
\item writing a new unpack scan
\item writing a new file specific scan
\item writing a new postrun scan
\end{itemize}

In the next course we will show how a database for the Binary Analysis Tool can be created.
}

\end{document}
